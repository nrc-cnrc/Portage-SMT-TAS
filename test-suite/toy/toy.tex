\documentclass[11pt]{article}
\usepackage{isolatin1}

\newcommand{\phs}{\tild{s}}   % source phrase
\newcommand{\pht}{\tild{t}}   % target phrase

\title{Running Portage: A Toy Example}
\date{}
\author{George Foster}

\begin{document}
\maketitle

\begin{center}
{~} \\ \tiny
   Groupe de technologies langagi{\`e}res interactives /
      Interactive Language Technologies Group \\
   Institut de technologie de l'information /
      Institute for Information Technology \\
   Conseil national de recherches Canada /
      National Research Council Canada \\
   Copyright \copyright 2006, Sa Majest{\'e} la Reine du Chef du Canada /
      Her Majesty in Right of Canada
\end{center}

\section{Introduction}

Portage can be viewed as a set of programs for turning a bilingual corpus into
a translation system. This document describes this process using a simple
example distributed with the Portage source code (in {\tt
  Portage/test-suite/ toy}). The example is for French to English translation,
using text from the Hansard corpus. It is too small for good translation, but
large enough to give the flavour of a more realistic setup. Running time is
about half an hour.

\subsection{Running the Example}

To begin, you must build or obtain the Portage executables and ensure that they
are in your path. Then {\tt cd} to the {\tt Portage/test-suite/toy} directory. All
commands described below are issued from this directory, and all files are
located there (including this document).

As you work through the example, the commands that you are supposed to type are
preceded by a prompt ({\tt >}), and the system's response is not. For instance,
the following indicates that you should type {\tt ls}, and shows the directory
listing that results:
\begin{verbatim}
   > ls
   canoe.ini          hans3_en.txt       toy.pdf
   hans1_en.txt       hans3_fr.txt       toy.tex
   hans1_fr.txt       Makefile           train_en.lm
   hans2_en.txt       rescore-model.ini  hans2_fr.txt       
\end{verbatim}
For the sake of brevity, the system's output is not always reproduced in full
in the examples. Also, results (especially numbers) may vary a little from the
ones shown, due to platform differences.

Many of the commands are expressed as {\tt make} targets. This has the
advantage of requiring less typing, while still allowing you to see the actual
commands executed by the system, which are always echoed by {\tt make} (and
which of course you could also type directly). {\tt make} also lets you skip
steps. For example, if you are not interested in any steps before decoder weight
optimization, you can go to section~\ref{COW} and type {\tt make cow} to begin
at that point (although you will have to wait until the system executes the
necessary commands to catch up). Here are some other useful make commands:
\begin{itemize}
\item {\tt make all} runs all remaining steps at any point.\footnote{Doing this
    at the beginning and comparing the results with the ones printed in
    section~\ref{TranslatingTesting} is a good way to check if your
    installation is correct.}
\item {\tt make clean} cleans up and returns the directory to its initial state

\item {\tt make -j} {\em anything} builds the target {\em anything} by running
      commands in parallel whenever possible. This can be much faster if you
      have a multiprocessor machine.
\end{itemize}

\subsection{Overview of the Process}

Here is an overview of the Portage process, as described in the following
sections (section numbers are in brackets):
\begin{enumerate}
\item Corpus preparation (\ref{CorpusPreparation}): includes tokenization
      (\ref{Tokenization}), alignment (\ref{Alignment}), lowercasing
      (\ref{Lowercasing}), and splitting (\ref{Splitting}).
\item Model training (\ref{Training}): includes language model (\ref{LM}),
      translation model (\ref{TM}), decoder weight optimization (\ref{COW}),
      and rescoring model training (\ref{RAT}).
    \item Translating and testing (\ref{TranslatingTesting}): includes
      translating (\ref{Translating}) and testing (\ref{Testing}).
\end{enumerate}
These steps are fairly standard, but of course there are many variants on the
simple process illustrated here. For more information, see the Portage HTML
user documentation. For detailed information on a particular program, run it
with the {\tt -h} flag.

\section{Corpus Preparation} \label{CorpusPreparation}

Corpus preparation involves converting raw document pairs into tokenized,
sentence-aligned files in the format expected by Portage. Once this done,
the corpus is split into different parts used for training and testing.

The input to this process is a set of bilingual file pairs in plain text
format, like the example files {\tt hans*.txt}.\footnote{Although
  this directory contains a set of Hansard text files, these are not hardcoded
  into the examples. Any other English/French parallel corpus can be
  substituted, provided the file pairs are named {\em file}{\tt\_en.txt}, {\em
    file}{\tt\_fr.txt}, and provided they contain only plain text or allowed
  markup as described below.}
 To help the alignment process,
it can be beneficial to leave some matching markup in these files to identify
section boundaries and other important structure. This markup should be
enclosed in angle brackets, like the markup codes {\tt <sect>}, {\tt <time
  1400>}, etc that appear in the {\tt hans*.txt} files. It is filtered out
during alignment and is not used in subsequent steps.

\subsection{Tokenization} \label{Tokenization}

The first step is to tokenize and sentence-split the plain text files. This
is done with the program {\tt tokenize.pl}, which needs to be told which
language to expect as input (currently either English or French):
\begin{verbatim}
   > make tokenize
   tokenize.pl -lang=en hans1_en.txt hans1_en.tok
   tokenize.pl -lang=fr hans1_fr.txt hans1_fr.tok
   tokenize.pl -lang=en hans2_en.txt hans2_en.tok
   tokenize.pl -lang=fr hans2_fr.txt hans2_fr.tok
   tokenize.pl -lang=en hans3_en.txt hans3_en.tok
   tokenize.pl -lang=fr hans3_fr.txt hans3_fr.tok
\end{verbatim}
Notice that the output (.tok) files now contain one sentence (or sentence-like
unit) per line, with individual tokens separated by blanks.

\subsection{Alignment} \label{Alignment}

Once the files are tokenized, they can be aligned to the sentence level using
the {\tt ssal} program. The -f flag filters out angle-bracket markup that
appears in either input file, which is not needed after it has been used to aid
alignment:
\begin{verbatim}
   > make align
   ssal -f hans1_en.tok hans1_fr.tok
   mv hans1_en.tok.al hans1_en.al
   mv hans1_fr.tok.al hans1_fr.al
   ssal -f hans2_en.tok hans2_fr.tok
   mv hans2_en.tok.al hans2_en.al
   mv hans2_fr.tok.al hans2_fr.al
   ssal -f hans3_en.tok hans3_fr.tok
   mv hans3_en.tok.al hans3_en.al
   mv hans3_fr.tok.al hans3_fr.al
\end{verbatim}
This produces an output (.al) file for each input file. Alignment is
represented by line-for-line correspondence, so the corresponding .al files
should have exactly the same number of lines:
\begin{verbatim}
   > wc -l *.al
   2031 hans1_en.al
   2031 hans1_fr.al
   3659 hans2_en.al
   3659 hans2_fr.al
   3584 hans3_en.al
   3584 hans3_fr.al
   18548 total
\end{verbatim}

\subsection{Lowercasing} \label{Lowercasing}

The last text-transformation step is to convert all files to lowercase, in
order to reduce data sparseness:
\begin{verbatim}
   > make lowercase
   lc-latin.pl hans1_en.al hans1_en.lc
   lc-latin.pl hans1_fr.al hans1_fr.lc
   lc-latin.pl hans2_en.al hans2_en.lc
   lc-latin.pl hans2_fr.al hans2_fr.lc
   lc-latin.pl hans3_en.al hans3_en.lc
   lc-latin.pl hans3_fr.al hans3_fr.lc
\end{verbatim}

\subsection{Splitting the Corpus} \label{Splitting}

The corpus must now be split into separate portions in order to run
experiments. Distinct, non-overlapping sub-corpora are required for model
training (see section~\ref{Training}), for optimizing decoder weights
(section~\ref{COW}), for training a rescoring model (section~\ref{RAT}), and
for testing (section~\ref{Testing}). Typically the latter three corpora are
around 1000 segments each, although in the example below we use 100 segments to
reduce running time. Also, if the corpus is chronological, then it is a good
idea to choose the test corpus from the most recent material, which is
likely to resemble future text more closely.

In this example, the files for training, optimizing decoder weights, rescoring
model training, and testing are called {\tt train}, {\tt dev1}, {\tt dev2}, and
{\tt test}.\footnote{{\em dev} stands for {\em development}---the name
often given to training steps that involve relatively few parameters.}
Splitting is performed by the script {\tt split-listed.pl}:
\begin{verbatim}
   > make split
   rm -f lines.en; touch lines.en
   echo "8974  train_en.al" >> lines.en
   echo "100 dev1_en.al" >> lines.en
   echo "100 dev2_en.al" >> lines.en
   echo "100 test_en.al" >> lines.en
   cat hans*_en.lc | split-listed.pl lines.en
   rm -f lines.fr; touch lines.fr
   echo "8974  train_fr.al" >> lines.fr
   echo "100 dev1_fr.al" >> lines.fr
   echo "100 dev2_fr.al" >> lines.fr
   echo "100 test_fr.al" >> lines.fr
   cat hans*_fr.lc | split-listed.pl lines.fr
\end{verbatim}
This creates files {\tt \{train,dev1,dev2,test\}\_\{en,fr\}.al}:
\begin{verbatim}
   > wc -l {train,dev1,dev2,test}_{en,fr}.al
   8974 train_en.al
   8974 train_fr.al
    100 dev1_en.al
    100 dev1_fr.al
    100 dev2_en.al
    100 dev2_fr.al
    100 test_en.al
    100 test_fr.al
  18548 total

\end{verbatim}

\section{Training} \label{Training}

This step creates various models and parameter files that are required for
translation. There are four steps in training: creating a language model,
creating translation models, optimizing decoder weights, and creating a
rescoring model.

\subsection{Creating a Language Model} \label{LM}

Portage does not come with programs for language model training. However,
it accepts models in the widely-used ``DARPA'' format which is produced by popular
toolkits such as CMU SLM (www.speech.cs.cmu.edu/ SLM\_info.html) or SRILM
(www.speech.sri.com/projects/srilm). Here is the SRILM command used to produce
the model {\tt train\_en.lm} supplied with this example:
\begin{verbatim}
   > make lm
   ngram-count -interpolate -kndiscount1 -kndiscount2 -kndiscount3 \
      -order 3 -text train_en.al -lm train_en.lm
\end{verbatim}
This assumes that the target language is English; for English to French
translation, a similar command could be used with {\tt train\_fr.al} to create
a French language model.

\subsection{Creating a Translation Model} \label{TM}

Creating a translation model involves two main steps: training IBM models in
both directions, then using them to extract phrase pairs from the corpus. These
steps are automated by the script {\tt train-phrases.sh}.  Because this script
expects its input to be isolated in a single directory, a preliminary step is
to create a temporary directory containing only the training corpus, as shown:
\footnote{If you have a multiprocessor machine, you can use {\tt
    train-phrases.sh -p} to run bidirectional IBM model training in parallel.}

\begin{verbatim}
   > make tm
   rm -rf tmp; mkdir tmp
   ln -s ../train_en.al tmp/train_en.al
   ln -s ../train_fr.al tmp/train_fr.al
   train-phrases.sh tmp
   train_ibm -v -s ibm1.fr_given_en ibm2.fr_given_en \
      tmp/train_en.al tmp/train_fr.al
   train_ibm -vr -s ibm1.en_given_fr ibm2.en_given_fr \
      tmp/train_en.al tmp/train_fr.al
   gen_phrase_tables -v -w1 -m8 -1 en -2 fr -ibm 2 \
      ibm2.fr_given_en ibm2.en_given_fr tmp/train_en.al tmp/train_fr.al
\end{verbatim}
The script writes log files containing information about IBM and
phrase training. For example, the IBM log files show convergence and pruning
statistics for each iteration:
\begin{verbatim}
   > egrep ppx log.train_ibm.en_given_fr
   Iter 1 (IBM1): prev ppx = 356.169, size = 1476280 word pairs
   iter 2 (IBM1): prev ppx = 109.36, size = 1355987 word pairs
   iter 3 (IBM1): prev ppx = 66.7163, size = 1191090 word pairs
   iter 4 (IBM1): prev ppx = 53.1018, size = 1026382 word pairs
   iter 5 (IBM1): prev ppx = 48.1907, size = 885809 word pairs
   iter 6 (IBM2): prev ppx = 46.0251, size = 885809 word pairs
   iter 7 (IBM2): prev ppx = 18.8685, size = 885809 word pairs
   iter 8 (IBM2): prev ppx = 13.7175, size = 885809 word pairs
   iter 9 (IBM2): prev ppx = 11.8744, size = 885809 word pairs
   iter 10 (IBM2): prev ppx = 11.071, size = 885809 word pairs
\end{verbatim}

The IBM models and phrasetables are written to files {\tt ibm[12].*} and {\tt
  phrases.*} respectively. The phrasetable {\tt phrases.fr2en} is the main
source of information for French to English translation. Each of its lines is
of the form:
\begin{verbatim}
   fr ||| en ||| p(fr|en) p(en|fr)
\end{verbatim}
where {\tt fr} is a French source phrase, {\tt en} is an English target phrase,
{\tt p(fr|en)} is the probability that {\tt fr} is the translation of {\tt en}, and
{\tt p(en|fr)} is the probability that {\tt en} is the translation of {\tt fr}. The
reverse phrasetable {\tt phrases.en2fr}, used for English to French
translation, contains the same information but with {\tt en} and {\tt fr}
reversed. Here is a sample line from these files:
\begin{verbatim}
   > select-line 89 < phrases.fr2en
   r�glements propos�s ||| proposed regulations ||| 0.5 1
   > select-line 89 < phrases.en2fr
   proposed regulations ||| r�glements propos�s ||| 1 0.5
\end{verbatim}
(This example is carefully chosen; because of the small size of the training
corpus, many other entries in the phrase table are incorrect.)

\subsection{Optimizing Decoder Weights} \label{COW}

The language and translation models are the main sources of information used
for translation, which is performed by the {\tt canoe} decoder. In order to get
reasonable translation quality, the weights on these (and other) sources of
information need to be tuned. The tuning process is carried out by a script
called {\tt cow.sh}, which runs {\tt canoe} several times. {\tt cow.sh}
requires a configuration file for {\tt canoe}, called {\tt canoe.ini} by
default:
\begin{verbatim}
   > cat canoe.ini
   [ttable-multi-prob] phrases.fr2en
   [lmodel-file] train_en.lm
   [ttable-limit] 30
   [ttable-threshold] 0.000
   [stack] 70
   [beam-threshold] 0.01
   [distortion-limit] 7
\end{verbatim}
This indicates that the phrasetable to be used for translation is \\ {\tt
  phrases.fr2en} and that the language model is {\tt train\_en.lm}. The other
parameters control the search strategy---they have been set so as to favour
speed over quality for the purposes of this example.

Besides the configuration file, the other main arguments required by {\tt
  cow.sh} are a directory in which to store temporary files, a source file, and
  one or more correct (ie human) translations of the source
  file.\footnote{This example uses only one translation, but when multiple
  alternative translations are available, it can be advantageous to use them.}
Here we call the temporary
directory {\tt foos}, and use the {\tt dev1} files for weight tuning:
\begin{verbatim}
   > make cow
   rm -rf foos; mkdir foos
   cow.sh -nbest-list-size 100 -maxiter 10 -filt -floor 2 \
      -workdir foos dev1_fr.al dev1_en.al >& log.cow
\end{verbatim}
Because {\tt cow.sh} takes a long time to run 
(about 25 minutes for this
example on a single machine at NRC)\footnote{Running time for {\tt cow.sh} can be
  reduced by executing it in parallel. Do {\tt cow.sh -h} to see documentation on
  this option.}
and writes large amounts of logging
information, it is usually a good idea to redirect all of its output to a log
file, as shown here; and also to run it in the background (not done here, for
compatibility with the {\tt make} procedure).  Progress can be monitored by
looking at the file {\tt rescore-results}, which
shows the translation quality (measured by BLEU score---see
section~\ref{Testing} for a description), as well as the current weights, for
each iteration. Note that the BLEU score does not necessarily increase
monotonically, as can be seen in several places in the {\tt rescore-results}
file obtained with this example:
\begin{verbatim}
> cat rescore-results
BLEU score: 0.157469   -d 1 -w 0 -lm 1 -tm 1
BLEU score: 0.166618   -d 0.1260017157 -w 0.4231236577 -lm 1 -tm 0.591891706
BLEU score: 0.134856   -d 0.3837988973 -w 0.6390675902 -lm 1 -tm 0.05515978858
BLEU score: 0.184703   -d 0.1502487957 -w 0.436016798 -lm 1 -tm 0.3171886504
BLEU score: 0.182674   -d 0.1501607597 -w -0.1386783123 -lm 1 -tm 0.2572171986
BLEU score: 0.185179   -d 0.1486719251 -w 0.4118739367 -lm 1 -tm 0.3253635168
BLEU score: 0.185113   -d 0.1496432126 -w 0.4299308062 -lm 1 -tm 0.3257338107
BLEU score: 0.185113   -d 0.1496805698 -w 0.4292495847 -lm 1 -tm 0.3257473707
BLEU score: 0.185113   -d 0.1493744254 -w 0.4283171892 -lm 1 -tm 0.3256304562
BLEU score: 0.185113   -d 0.1496380866 -w 0.4294756055 -lm 1 -tm 0.3257317543
\end{verbatim}

The output from {\tt cow.sh} is written to the file {\tt canoe.ini.cow}
(assuming the original configuration file was called {\tt canoe.ini}). This
duplicates the contents of {\tt canoe.ini}, but adds the optimal weights
learned on the development corpus:
\begin{verbatim}
   > cat canoe.ini.cow
   [ttable-file-t2s] phrases.en_given_fr
   [ttable-file-s2t] phrases.fr_given_en
   [lmodel-file] train_en.lm
   [weight-d] 0.1486719251
   [weight-w] 0.4118739367
   [weight-l] 1
   [weight-t] 0.3253635168
   [ttable-limit] 30
   [ttable-threshold] 0
   [stack] 70
   [beam-threshold] 0.01
   [distortion-limit] 7
\end{verbatim}
where the {\tt weight-} parameters pertain to, respectively, the distortion
model, the word-length penalty, the language model, and the translation
model. The language model usually obtains the highest weight, as in this case. 

\subsection{Training a Rescoring Model} \label{RAT}

The final training step is to create a model for rescoring nbest
lists. Rescoring means having {\tt canoe} generate a list of $n$ (typically
1000) translation hypotheses for each source sentence, then choosing the best
translations from among these. The advantage of this procedure is that the
choice can be made on the basis of information that is too expensive for {\tt
canoe} to use during search. This step usually gives a modest improvement over
the results obtained using {\tt canoe} alone.

Training a rescoring model involves generating nbest lists, then calculating
the values of selected {\em features} for each hypothesis in each list. A
feature is any real-valued function that is intended to capture the relation
between a source sentence and a translation hypothesis. A rescoring model
consists of a vector of feature weights set so as to optimize translation
performance when a weighted combination of feature values is used to reorder
the nbest lists.

Training is carried out by the {\tt rat.sh} script. This takes as input a
rescoring model that specifies which features to use, and it returns optimal
weights for these features. The model supplied with this example contains a
small set of useful features:
\begin{verbatim}
   > cat rescore-model.ini
   FileFF:ffvals,1
   FileFF:ffvals,2
   FileFF:ffvals,3
   FileFF:ffvals,4
   FileFF:ff.IBM2TgtGivenSrc.ibm2.en_given_fr
   FileFF:ff.IBM2SrcGivenTgt.ibm2.fr_given_en
   FileFF:ff.IBM1WTransTgtGivenSrc.ibm2.en_given_fr
   FileFF:ff.IBM1WTransSrcGivenTgt.ibm2.fr_given_en
   LengthFF
\end{verbatim}
Notice that most of the feature specifications begin with {\tt FileFF}; this
indicates that they are to be pre-calculated and stored in files, rather than
being calculated on the fly from the contents of the nbest lists. There are two
kinds of {\tt FileFF} features:
\begin{itemize}
\item Those that end with {\tt ffvals,}$i$ tell {\tt rat.sh} to use the
$i$\/th feature generated by the {\tt canoe} decoder itself. It is standard
practice to use all decoder features when rescoring, as is done
here.\footnote{Of course, it is is possible to use more than 4 features when
decoding. For large sets of decoder features, the command {\tt configtool
rescore-model:ffvals canoe.ini} is handy: it automatically extracts
features from {\tt canoe.ini}, and produces rescore-model entries for them.}

\item Those that end with {\tt ff.}{\em fname.args} tell {\tt
rat.sh} to generate values for the feature {\em fname} using arguments {\em
args}. For example,\\ {\tt FileFF:ff.IBM2TgtGivenSrc.ibm2.en\_given\_ch} says to
calculate the feature {\tt IBM2TgtGivenSrc} using the IBM model {\tt
ibm2.en\_given\_ch} which we trained earlier. To see a list of all
available features, type {\tt rescore\_train -H}.
\end{itemize}

Apart from the rescoring model, {\tt rat.sh} needs a source file and one or
more alternative translations for it (same as {\tt cow.sh}). These may be the
same files used for {\tt cow.sh}, but it is generally better to use different
ones, so here we use {\tt dev2}:
\begin{verbatim}
   > make rat
   cp -f rescore-model.ini rescore-model
   rat.sh -n 1 train -n 1 -f canoe.ini.cow -K 100 rescore-model \
      dev2_fr.al dev2_en.al >& log.rat
\end{verbatim}
As with {\tt cow.sh}, it is usually preferable to run {\tt rat.sh} in the
background, unlike what is shown here. The {\tt -n 1} options cause {\tt
rat.sh} to run on only one processor; to run in parallel, simply omit these.

The output from {\tt rat.sh} is written to the file {\tt rescore-model}:
\begin{verbatim}
   > cat rescore-model
   FileFF:ffvals,1 0.8429076672
   FileFF:ffvals,2 -0.5938655734
   FileFF:ffvals,3 1
   FileFF:ffvals,4 0.3042182028
   FileFF:ff.IBM2TgtGivenSrc.ibm2.en_given_fr 0.1957350224
   FileFF:ff.IBM2SrcGivenTgt.ibm2.fr_given_en -0.02397140861
   FileFF:ff.IBM1WTransTgtGivenSrc.ibm2.en_given_fr 0.03992729634
   FileFF:ff.IBM1WTransSrcGivenTgt.ibm2.fr_given_en 0.0126716774
   LengthFF 0.1795582324
\end{verbatim}
This is identical to {\tt rescore-model.ini}, except that each feature is now
assigned a weight. Other by-products created by {\tt rat.sh} include the nbest
lists {\tt dev2\_fr.al.nbest.gz}, and the corresponding decoder features \\{\tt
  dev2\_fr.al.ffvals.gz} and additional features {\tt dev2\_fr.al.ff.*}. All of
these files are compressed to save space---this generates some harmless
warnings later in the process.

\section{Translating and Testing} \label{TranslatingTesting}

\subsection{Translating} \label{Translating}

Once training is complete, the system can be used to translate.  As mentioned
above, there are two options for translating. The simplest is to decode using
the configuration file produced by {\tt cow.sh}:
\begin{verbatim}
   > make translate
   canoe -f canoe.ini.cow < test_fr.al > canoe_ON_test_fr.out
\end{verbatim}
This produces the output file {\tt canoe\_ON\_test\_fr.out}, containing one
line for each source line in {\tt test\_fr.al}.

The other option for translating is to generate nbest lists and rescore them
using the model generated in section~\ref{RAT}. To do this, we use the {\tt
rat.sh} script again, but this time in translation mode:
\begin{verbatim}
   > make rescore
   rat.sh -n 1 trans -n 1 -f canoe.ini.cow -K 100 rescore-model test_fr.al
   mv test_fr.al.rat rescore_ON_test_fr.out
\end{verbatim}
This produces the output file {\tt rescore\_ON\_test\_fr.out}.

Notice that the output files contain only lowercase, tokenized text. A
postprocessing step is required to restore normal letter case and to undo the
tokenization. Postprocessing is beyond the scope of this example, since case
restoration software is not supplied with Portage.

\subsection{Testing} \label{Testing}

Translation quality can be evaluated using BLEU score, which is a measure of
how well the translation matches an alternative translation that is known to be
correct. It is based on the number of short word sequences that the two
translations have in common, and varies between 0 for no matches to 1 for a
perfect match. BLEU is calculated by the program {\tt bleumain}:
\begin{verbatim}
   > make test
   bleumain canoe_ON_test_fr.out test_en.al > canoe_ON_test_fr.bleu
   bleumain rescore_ON_test_fr.out test_en.al > rescore_ON_test_fr.bleu
\end{verbatim}

The output from {\tt bleumain} contains match statistics of various orders,
followed by the global BLEU score:
\begin{verbatim}
   > cat canoe_ON_test_fr.bleu
   1-gram (match/total) 1211/2360 0.513136
   2-gram (match/total) 453/2260 0.200442
   3-gram (match/total) 204/2161 0.0944007
   4-gram (match/total) 113/2062 0.0548012
   Sentence length: 2360; Best-match sentence length: 2378
   Score: -1.8923
   BLEU score: 0.150725

   > cat rescore_ON_test_fr.bleu
   1-gram (match/total) 1225/2373 0.516224
   2-gram (match/total) 463/2273 0.203696
   3-gram (match/total) 215/2174 0.098896
   4-gram (match/total) 114/2075 0.0549398
   Sentence length: 2373; Best-match sentence length: 2378
   Score: -1.86899
   BLEU score: 0.154279
\end{verbatim}
These results indicate that the translation produced by rescoring is slightly
better than the one produced by plain decoding. Given the small size of the
test set, it seems unlikely that the difference is statistically
significant. To test this hypothesis, we can use {\tt bleucompare}, which does
a comparison using pairwise bootstrap resampling:
\begin{verbatim}
   > bleucompare rescore_ON_test_fr.out canoe_ON_test_fr.out \
      REFS test_en.al
   rescore_ON_test_fr.out got max bleu score in 81.5% of samples
   canoe_ON_test_fr.out got max bleu score in 18.5% of samples
\end{verbatim}
This indicates that the difference is only significant at the 80\% level, which
is too low to be acceptable.

\end{document}
