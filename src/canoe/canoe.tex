\documentclass[12pt]{amsart} \usepackage{epsfig} \usepackage{rotating}

\newcommand{\file}{\textbf} \newcommand{\url}{\underline}
\newcommand{\code}{\small\texttt}

\begin{document}

\title{Canoe} \maketitle

{\tiny
\noindent
Canoe\\ Aaron Tikuisis\\ Groupe de technologies langagi{\`e}res interactives /
Interactive Language Technologies Group\\ Institut de technologie de
l'information / Institute for Information Technology\\ Conseil national de
recherches Canada / National Research Council Canada\\ Copyright 2005/2006,
Conseil national de recherches Canada / National Research Council Canada }
\\

\noindent
Please keep this document up to date whenever you make changes to the code!

{\tiny
\noindent
Version info: \verb+$Id$+
}
\\

Canoe is a phrase-based decoder, replicating and extending the
algorithm described by Koehn \cite{koehn03}.  This document
describes the modular organization of Canoe and goes into some detail on
certail parts of the algorithm.  A general convention used here and throughout
the code and documentation is that the ``source" text refers to the input to
the translator while the ``target" text refers to the output.

It may be noticed that I often use abstract classes to represent parts used in
the decoder, even when there is only one conceivable reasonably implementation.
This is out of caution, since limits on what is conceivable can change.  As a
case in point, specialized implementations of some of these classes are often
used to test other classes, allowing more specialized component testing.

\begin{sidewaysfigure}
\centerline{\epsfig{figure=uml.eps}}
\caption{Interactions between Canoe Modules\label{fig:uml}}
\end{sidewaysfigure}

\section{Decoding Algorithm}
The core of the decoding algorithm is found in \file{decoder.cc}.  In
pseudo-code, it is as follows: \\ {\noindent \code{
\noindent
Let $F = $ length of the source sentence\\ Initialize $(F + 1)$ hypothesis
stacks, $H[0], \dots H[F]$\\ Put the initial state (an empty partial
translation) into $H[0]$\\ For $i = 0$ to $F - 1$\\ \indent For $state \in
H[i]$\\ \indent \indent Let $P = $ the set of phrase options that may be added
to $state$\\ \indent \indent For $phrase \in P$\\ \indent \indent \indent Let
$state' = state + phrase$\\ \indent \indent \indent Let $n = $ the number of
source words covered by $state'$\\ \indent \indent \indent Add $state'$ to
$H[n]$, recombining and pruning as necessary } }

Many details are abstracted away in this presentation of the algorithm.  More
explicitly, the following objects are required by the algorithm (some of which
are obviously needed, others are more implicit):

\begin{itemize}

\item
Hypothesis stacks.

\item
States, representing partial translations.

\item
Phrase options.

\item
Something to find the set of phrase options that may be added to a state
(Phrase Finder).

\item
Something to compute the probability and estimated future probability of a
partial translation (Phrase Decoder Model).

\end{itemize}

Classes for these objects make up the core of Canoe.  Each one is described in
detail below; first is a description of some basic structures used by these
more complex classes.


\section{Basic Structures}
These basic structures, except for \code{Uint}, are defined in
\file{canoe\_general.h}.

\begin{itemize}

\item {\bf Target words} \par A words in the target vocabulary are stored as
\code{Uint}.  The Phrase Decoder Model implemented (described below) uses
the \code{Voc} class in \code{utils} to convert these to and from the string
representation as necessary.

\item {\bf Target phrases} \par A target phrase (\code{Phrase}) is represented
by a \code{vector} of target words.

\item {\bf Source words} \par Since the decoder algorithm works on one source
sentence at a time, the source sentence is generally considered to be fixed.
Source words are thus referred to by their index (a \code{Uint}) in the source
sentence, beginning at 0.

For most purposes, a source word is thought of as a source word in context, so
there is no cause for concern if a word appears multiple times in the same
source sentence.

\item {\bf Source phrases} \par Again, since the source sentence is fixed,
source phrases are limited to subphrases of this sentence.  A source phrase is
represented by a range (\code{Range}) $[a, b)$, where $a$ is the index of the
first word in the phrase and $b$ is the index of the last word in the phrase
plus one.  The notation $[a, b)$ used to represent a range is meant to be
suggestive of the underlying semantics.

\item {\bf Sets of source words} \par A set of source words (\code{UintSet}) is
represented as a \code{vector} containing source phrases that cover the set,
with the following conditions on the \code{vector}:

\begin{itemize}
\item
The \textbf{number} of source phrases is required to be minimal (equivalently,
the ranges should not overlap or touch; eg. $[1, 2), [2, 3)$ is not allowed,
instead $[1, 3)$ should be used).
\item
The ranges are required to be in increasing order (due to the first condition,
this statement means the same thing regardless of whether the ordering is by
left end-point or right end-point).
\end{itemize}

\end{itemize}


\section{Phrase Options}
The \code{PhraseInfo} structure, defined in \file{phrasedecoder\_model.h},
represents a phrase option, which is a pair (source phrase, target phrase)
along with an associated score.

\code{MultiTransPhraseInfo}, in \file{basicmodel.\{h,cc\}}, is a subclass of
\code{PhraseInfo}, used to allow multiple scores from different translation
models.


\section{Phrase Decoder Model}

\subsection{PhraseDecoderModel Abstract Class}
The \code{PhraseDecoderModel} object represents the model describing the
probability space of target sentences, with the source sentence considered
fixed.  Additional related information about the source sentence and the space
of target sentences is also available through this object.

In particular, the following functions are provided:

\begin{itemize}
\item
Conversion from a source phrase represented by a \code{vector} of \code{Uint}'s
to a \code{string}.

\item
Getting the number of words in the source sentence.

\item
Getting all the phrase options.

\item
Computing the marginal score of a partial translation, ``marginal" meaning the
difference between the score of the partial translation and its back reference.

\item
Computing the future score of a partial translation.

\item
Determining whether two partial translations may be recombined.

\item
Computing a hash value for partial translations, such that two partial
translations that may be recombined will have the same hash value.
\end{itemize}

The class \code{PhraseDecoderModel}, defined in \file{phrasedecoder\_model.h},
is entirely abstract.  \code{BasicModel}, in \file{basicmodel.\{h,cc\}}, is the
implementation of \code{PhraseDecoderModel} used by Canoe.

\subsection{BasicModel Class}
The class \code{BasicModel} represents a log-linear model, using the following
as feature functions:

\begin{itemize}

\item
Some number (at least one) phrase translation model.

\item
Some number of language models.  Although it is
permissible to use no language models, it is not recommended, especially in
light of the fact that if zero language models are used, an optimal translation
can be found using a much simpler and efficient algorithm.

\item
Distortion, a non-negative integer-valued feature defined as follows: \\ Let
$s_1, \dots , s_n$ be the source phrases, arranged in order as the
corresponding target phrases appear in the target sentence. \\ Let $b_i$ be the
beginning index of the source phrase $s_i$, $b_0 = 0$. \\ Let $e_i$ be the end
index of the source phrase $s_i$ (the last index in the phrase plus $1$),
$e_{n+1} = $ the total number of words in the source sentence. \\ Then
distortion is $-\sum_{i = 0}^n |b_{i+1} - e_i|$. Note that this differs from
Koehn's definition of distortion, since Koehn's distortion doesn't include
the distortion from the last phrase to the end of the sentence ($|b_{n+1} -
e_n|$).\\[.2cm]
Note (GF): Distortion is now calculated by the \code{DistortionModel} abstract
class, which is passed as a parameter to \code{BasicModel}. The default
behaviour described here is implemented in the  \code{WordDisplacement}
model. For a list of alternative models, see \code{DistortionModel::create()}
in \file{distortionmodel.cc}.
\\[.2cm]

\item
Number of words in the target sentence, calculated by the \code{LengthFeature}
class.

\item
Segmentation model, calculated by one of the subclasses of
\code{SegmentationModel} found in \file{segmentmodel.h}.

\item
Any new feature you write.  You can subclass DecoderFeature to add your new
experimental features.  Make sure you read the interface documentation in
\file{decoder\_feature.h} carefully and implement all necessary virtual
functions.  Doing this incorrectly will likely make your experiments invalid,
so take time to understand this interface and the functionning of the decoder
well before you implement your own features.

\end{itemize}

\subsection{Future Score Computation}

Based on the A* search and the beam search concepts\footnote{Technically, our
future score is not an admissible heuristic, but it works reasonably well
anyway.}, the future score is used to
try to reduce the chance of pruning partial translations that shouldn't be
pruned because they lead to optimal final translations.  An attempt is made to
incorporate as much information as possible into the future score computation
while allowing it to be computed reasonably quickly and avoiding any artificial
bias.

The future score is thus defined as the maximum possible $($translation model
score $+$ unigram language model score $+$ length score $+$ distortion score
$+$ segmentation score $)$
for translating the rest of the source words (where each score is already
weighted).  Hence, the future score takes into account everything except the
complete language model score.  I avoided using the trigram language model
score for phrases, out of caution that it may unfairly score phrases; however,
this has not been thoroughly considered and it may be beneficial to change.

The translation model score, unigram language model score, and length score can
be computed on a phrase-by-phrase basis, and using dynamic programming, the
maximum of this combined score is computed at the outset for each source phrase
(as Koehn does).

It is not difficult to see that the best possible distortion score is
achievable by translating the remaining source phrases from first (least index)
to last(greatest index).

In the future score computation function, it is also checked if it is possible
to complete the translation without violating a hard limit on distortion
distance between two phrases.  While I am confident that an if-and-only-if
condition exists that can be checked in $O($number of source ranges not
covered$)$ time, I have not put enough thought into this to determine exactly
what such a condition would be.

Instead, the following condition is checked which may err on the side of being
too conservative: The distance from the last (most recent) source word
translated to the first (least index) source word not translated must be at
most the distortion limit AND the distance between any two consecutive ranges
of source words not covered must be at most the distortion limit.

While it is easy to see that the second part of this condition is necessary,
the following example shows how the first part of this condition is too
cautious.  Suppose a distortion limit of $3$ and consider a partial hypothesis
in which only the fourth source word (of $5$) has been translated.  Using $-$
to represent untranslated words, $*$ to represent translated words, and $.$ to
represent the current position, this state is written $---*.-$.  Since the
distance from the current position to the start of the sentence is $4$, this
partial translation fails the above condition.  However, the following is a
valid sequence of translations without violating the distortion limit:
\begin{align*}
-*.-*- \; &\text{(distortion $3$)} \\ *.*-*- \; &\text{(distortion $2$)} \\
***.*- \; &\text{(distortion $1$)} \\ *****. \; &\text{(distortion $1$)}
\end{align*}

Finding a tighter condition does not seem to be an important priority, since
the only exceptions involve very unusual translation sequences.

\subsection{Creating a BasicModel}
\code{BasicModelGenerator} is a class used in order to create a
\code{BasicModel}.  By having a seperate class that creates the model objects,
the translation model(s) and language model(s) can be loaded once and reused
for multiple sentences.  Different forms of the constructor allow the model
files to be loaded with the source sentences known in advance or not.  If the
source sentences are known in advance, only the relevent parts of the
translation model(s) are loaded, thus potentially saving time and space.
However, this is given as an option since in some circumstances the source
sentences are not known in advance.

Many of the functions of \code{BasicModel} are implemented by deferring to an
appropriate function of \code{BasicModelGenerator}.

\subsection{Finding (All) Phrase Options}
When the \code{BasicModelGenerator} creates a \code{BasicModel}, it must find
applicable phrase translation options.  The class \code{PhraseTable}, in
\file{phrasetable.\{h,cc\}}, picks phrase options from the phrase table(s)
given and does pruning using weighted scores.

When multiple phrase tables are given, the pool of phrase options is the union
of the options over all the tables.  However, when a phrase translation is not
given in some table, it is assigned a very low probability
(\code{LOG\_ALMOST\_0}) for that model; thus, depending on the pruning
threshold, the phrase options used may actually be closer to the intersection.

Under two different circumstances, \code{BasicModelGenerator} may add
additional phrase translation options:

\begin{itemize}
\item
First, marked translations may be specified to the model.  The score for marked
translations is multiplied by the sum of the translation model weights; for
example, if a mark is given with log probability of $-1.0$, and two translation
models are given with weights $0.5$ and $0.7$ respectively, then the mark is
used as a translation option with score $-1.2$, not $-1.0$.

\item
Second, after all other phrase options have been added, provisions are made for
source words that have no translation options.

Supposing that no translation options exist for some given source word; then
unless a translation option exists and can be used for a phrase containing that
word, the decoder algorithm will result in an empty final hypothesis stack.  To
avoid this situation, any source word that has no phrase translation option is
given itself as a translation option, but with very low score
(\code{LOG\_ALMOST\_0}).  The low score allows that if a phrase containing the
given word can be translated, it probably will be instead of using this default
translation.
\end{itemize}

\subsection{Language Model Smoothing}
After finding all phrase translation options, it is checked that every word of
every option is in the language model.  Words that are not in the language
model are considered to have a very low unigram probability
(\code{LOG\_ALMOST\_0}) and (by mathematical necessity), a back-off weight of
$0$.

\subsection{Backwards Model}
\code{BackwardsModelGenerator} is a subclass of \code{BasicModelGenerator} used
to create models that make the decoder build the target sentence backwards
instead of forwards.  In terms of the implementation, the difference between a
backwards model and a forwards model is very superficial; thus, the entire
\code{BasicModel} class is reused here, and few functions in
\code{BasicModelGenerator} are overridden by \code{BackwardsModelGenerator}.

The backwards model is meant to be used with a language model(s) trained on a
corpus of backwards sentences, but a normal translation model.


\section{Partial Translations and Decoder States}
The \code{DecoderState} class represents an intermediary stage of the decoding
process, akin to what Koehn calls an hypothesis or search state.  The pieces of
information required to be in the decoder state are divided into ones that seem
essentially part of the partial translation and ones that are auxiliary and
necessary only for the decoding.

Specifically, a decoder state consists of:

\begin{itemize}
\item A partial translation
\item A score
\item An estimated future score
\item Recombined states
\item A reference to the previous state
\end{itemize}

Whereas the partial translation consists of:

\begin{itemize}
\item The last (most recent) phrase option
\item The set of source words not covered
\item The number of source words covered
\item A reference to the previous partial translation
\end{itemize}

The recombined states in \code{DecoderState} is stored of a \code{vector} of
\code{DecoderState}'s; however, the \code{DecoderState}'s in this vector should
not have any recombined states of their own.

\code{DecoderState} is declared in \file{decoder.h} and implemented in
\file{decoderstate.cc}; this implementation includes a function to create new
\code{DecoderState}'s by building on existing ones.  \code{PartialTranslation}
is declared in \file{phrasedecoder\_model.h} and implemented in
\file{partialtranslation.cc}.

\section{Hypothesis Stack}
\code{HypothesisStack} is an abstract class responsible for, at the least,
storing decoder states.

The \code{RecombHypStack} subclass uses a hash table to track recombinable
states, using \code{PhraseDecoderModel::computeRecombHash()} as the hash
function.  When two states are recombinable, the one with the higher estimated
future score is kept directly in the hypothesis stack and the one with the
lower score is added to the recombined states of the one with the higher score.

\code{HistogramThresholdHypStack}, a subclass of \code{RecombHypStack}, also
does histogram- and relative-threshold-pruning.

All of these classes are found in \file{hypothesisstack.\{h,cc\}}.

\section{Phrase Finder}
\code{PhraseFinder} is a class used to select from all the phrase options the
subset of options that may be added to a given partial translation.  Again it
is an abstract class, with only one subclass implemented:
\code{RangePhraseFinder}.

Two factors limit which options are selected:

\begin{itemize}
\item The source words in a phrase to be added must be a subset of the source
words not yet translated.
\item A hard limit may be set on the distortion distance from the end of the
last source phrase to the beginning of ones to be added.
\end{itemize}


\section{Source Sentence Input}
Since the source text has a non-trivial format (namely due to the XML mark-up),
the class \code{DocumentReader} was created to handle input text reading and
parsing.  It can be found in \file{inputparser.\{h,cc\}}.

\subsection{Mark-up Format}
A mark-up in the input must have the following format:\\ $<$\textit{MARKNAME}
target = ``\textit{TGTPHRASE}$(|$\textit{TGTPHRASE}$)$*'' $[$prob =
``\textit{PROB}$(|$\textit{PROB}$)$*''$]$ $>$ \textit{SRCPHRASE} $<$/
\textit{MARKNAME} $>$, \\ with the following exceptions and constraints:
\begin{itemize}
\item Each space may be replaced with any amount of whitespace.
\item \textbf{target} may be replaced by \textbf{english} (for compatibility
with Koehn's format).
\item If \textbf{prob} occurs, there must be as many \textit{PROB}'s as there
are \textit{TGTPHRASE}'s.
\item The \textit{MARKNAME}'s must match.
\item If the mark is not the first thing on the line, there must be a space
between the preceeding word and the opening $<$.
\item There may be no space between the opening $<$ and \textit{MARKNAME} or
between $<$ and / in the closing tag.
\end{itemize}

Instances of $<$ and $>$ that occur in the text (not as part of a mark) must be
escaped by $\backslash$.  If they occur in the text followed by whitespace or
not preceeded by whitespace, they will be accepted as regular characters, but
with a warning.

\section{N-Best Output}
The files \file{nbesttranslation.\{h,cc\}} contain our implementation, written
by Howard Johnson, of the Mohri and Riley algorithm for finding the N best
paths through a lattice \cite{mohri02}.   This code is invoked when the
\code{-nbest} option is used.  The best documentation for this code is the
paper itself.  The main function is written to mimic very closely the
pseudo-code there, and won't make sense if you don't read the paper first.

\section{Word Graph Output}
NB: we typically no longer use this functionality, since the n-best lists are
now directly extracted from Canoe's internal representation of the lattice (see
above).  Lattice output is still available in case you wish to do your own
lattice analysis.  The lattices are typically huge, though, so this is a very
inefficient process.  If you can add your analysis tools to canoe itself, using
our internal representation, you will get code that is much faster.

The files \file{wordgraph.\{h,cc\}} contain definitions and implementations of
functions to produce output representing the word graph produced by a decoding
run. Essentially this converts the internal state graph into a lattice that can
be processed by external programs (to produce analysis not supported directly
in canoe).

The internal state graph is a DAG in which states represent scored partial
translations. Each normal state has a back-link to a previous state
(representing some prefix of its partial translation), plus a list of
``recombined'' states representing partial translations that the model cannot
distinguish from the current one (same last two target words, set of covered
source words, and end position for most-recent source phrase).  Recombined
states always have lower scores than the normal states they are associated
with. Their back links point only to normal states, and they have no recombined
state list of their own.

The algorithm for printing the word graph as a lattice does a post-order
traversal of the graph, starting at the end states but outputting earlier
states first. The actual output takes the form of arcs connecting pairs of
states, which are labelled with the phrase associated with the destination
state. For normal states, the score on an arc is (the exponential of) the
difference between the log probability of the destination state and the source
state. For recombined source states, the arc score is (the exponential of) the
difference between the log probability of the destination state and the normal
source state they are associated with; this is because the destination state is
an extension of the normal state (only), and its probability reflects this.

An optional parameter to the word graph printing function is a reference to an
object of type \code{PrintFunc}.  This abstract functor class is used to
produce a string representation of an arc incident on a given decoder state.
There are three concrete implementations of \code{PrintFunc}:

\begin{itemize}
\item
\code{PrintPhraseOnly} is used to produce only the target phrase,
surrounded with double quotes, with any existing
double quotes escaped. For example, the phrase 
\verb/the " baddest " actor/ would would be converted to \\
\verb/"the \" baddest \" actor"/. Existing backquotes are escaped the same
way, so \verb/x \ y/ would become \verb/"x \\ y"/.

\item
Warning: this format has changed slightly.

\code{PrintTrace} produces trace output: the target phrase, the phrase
translation probability, and the source range (0-based start and end
indexes). For example, if the phrase above were the translation of the 3rd
to the 5th source word, inclusive, with probability 0.5, the resulting
representation would be:
\begin{verbatim}
"the \" baddest \" actor" |0.5|2|4|
\end{verbatim}
Encoding all of this information on a single arc requires surrounding
everything with quotes, and escaping the ``internal'' phrase quotes and their
existing backslashes, giving:
\begin{verbatim}
"\"the \\\" baddest \\\" actor\" |0.5|2|4|"
\end{verbatim}

\item
Warning: this format has changed slightly.

\code{PrintFFVals} produces output containing the target phrase and the values
of each feature function (this works only when the \code{PhraseDecoderModel}
used is of type \code{BasicModel}). For example (assuming only one language and
one translation model):
\begin{verbatim}
"\"the \\\" baddest \\\" actor\" (-1, -1, -5.95, -0.69)"
\end{verbatim}
where the scores are, respectively: distortion, word penalty, language model,
and translation model (the number of significant digits has been shortened for
to make it fit the page better). If more than one LM and TM are being used, all
LM scores are written before all TM scores.
\end{itemize}
\code{PrintFunc} and its subclasses just mentioned are also found in
\file{wordgraph.\{h,cc\}}.

The word graph used to have an option for pruning out states that are not on
some path having probability greater than a given threshold.  The algorithm
basically kept track of the score of the current path to the final state
(during a postorder traversal), and combined this with the score of the current
state to see if a complete path with a high enough probability existed. If it
did, it would print out the main transition to the state, along with
transitions from any recombined states that had high enough probabilities.

The problem with this is that including states on some path that exceeds the
probability threshold will in general also permit paths that {\em don't} exceed
the probability threshold. So although some low-probability paths can be weeded
out by this technique (ie, ones that don't share states with high-probability
paths), many (probably most) can't. In practice, the small reduction in the
size of the lattice didn't seem worth the extra complexity and the unprinicpled
discrimination among paths. The pruning feature was therefore disabled (by GF,
Oct 2005); however, due to a bug in the way the algorithm had been previously
implemented, there was no observed change in lattice or nbest output.

\section{Main Function}
The main method of the Canoe program is implemented in \file{canoe.cc}.  The
command-line interface to Canoe is fairly similar to that described by Koehn;
for a detailed description of the command-line interface, use the -h option
with Canoe.

\bibliographystyle{plain} \bibliography{canoe}

\end{document}
