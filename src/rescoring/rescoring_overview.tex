\documentclass[12pt]{amsart}
                                %\usepackage{astron}

\newcommand{\argmax}{\mathop{\rm argmax}}
                                %\newcommand{\score}{\mathop{\rm score}}

\newcommand{\file}{\textbf} \newcommand{\url}{\underline}
\newcommand{\code}[1]{{\small \texttt{#1}}}

\begin{document}

\title{Overview of Code} \maketitle

{\tiny
  \noindent K-Best Rescoring Module\\ Aaron Tikuisis\\ Groupe de technologies
  langagi{\`e}res interactives / Interactive Language Technologies Group\\ Institut
  de technologie de l'information / Institute for Information Technology\\
  Conseil national de recherches Canada / National Research Council Canada\\
  Copyright \copyright 2005, 2006, Sa Majest{\'e} la Reine du Chef du Canada /
     Her Majesty in Right of Canada
}

\noindent
Please keep this document up to date whenever you make changes to the code!

{\tiny
\noindent
Version info: \verb+$Id$+
}
\\

\section*{Powell's Algorithm}

Originally, I got Powell's algorithm from \cite{NR-C++}.  I have made minor
modifications to the code (obviously, with no change to the actual algorithm),
and my version is in \file{powell.cpp}.  The changes made to the code are
as follows:

\begin{enumerate}
\item Instead of using the NR implementation of arrays and vectors, we use the
      uBLAS implementation in boost \cite{uBLAS}.

\item The original code performed minimization, which was changed to
      maximization.

\item The call to a generic line maximization method (also in the NR code
      library) is replaced by a call to the line maximization method specific
      to this problem (which I call Och's algorithm).
\end{enumerate}

The function signature for Powell's algorithm, \code{powell()}, can be found in
the header file \file{powell.h}, where the parameters are explained thoroughly.

The implementation of this function depends only on Och's algorithm (below),
and naturally on the class used as ScoreStats in the template.  A simple
program to test this function is contained in \file{testpowell.cpp}.

A comprehensive explanation of Powell's algorithm can be found at
\url{http://www.library.cornell.edu/nr/bookcpdf/c10-5.pdf} (sample pages from
\cite{NR-C}).

\section*{Och's Line Maximization Algorithm}

I implemented the line maximization method described in \cite{och-linemin}, in
order to be used by Powell's algorithm.  This code is found in
\file{linemax.cpp}

The function signature for Och's line maximization, \code{linemax()}, can be found in the header file \file{linemax.h}, where the parameters are explained thoroughly.

The implementation of this function does not depend on any other code here
except naturally on the class used as ScoreStats in the template.  A simple
program to test this function is contained in \file{testlinemax.cpp}.

A thorough explanation of the algorithm can be found in the file
\file{ochlinemax.pdf}.

\section*{Translation Quality Metrics}

Och's algorithm requires a class representing values used to compute a metric
for translation quality.  The function signatures for Powell's algorithm and
Och's algorithm use a template for this class; however, the following interface
is required: \\ \code{
  class ScoreStats {\indent ScoreStats(); \\
    \indent score(); \\ \indent output(ostream \&out); \\
  } \\ ScoreStats operator-(const ScoreStats \&s1, const ScoreStats \&s2); \\
  ScoreStats operator+(const ScoreStats \&s1, const ScoreStats \&s2);
}

A couple classes have been designed to be used in place of \code{ScoreStats}:
\begin{itemize}
\item A simple, all-purpose, additive metric is found in the class
      \code{LinearMetric}, in \file{linearmetric.cpp}.
  The score is provided as a parameter to the constructor.
\item A \code{BLEUstats} object represents the collection of BLEU-related
      statistics for one or a set of candidate translation sentences.
  Given a candidate translation and a set of reference translations (as
  strings), the statistics are computed in the BLEUstats constructor.  This
  class is defined in \file{bleu.h}, in the eval module.  The details of the
  BLEU metric for evaluating translation quality is described in detail in
  \cite{bleu}.  The program only does training using the BLEU score as a
  metric.
\end{itemize}

\section*{Feature Functions}

A number of feature functions (evaluated on a source-target sentence pair) are
an intrinsic part of the statistical model used here.  A fairly simple abstract
class was created to represent a feature function, and a few relevent feature
functions have been implemented so far.

The abstract class is, for obvious reasons, called \code{FeatureFunction}, and
is defined in the header file \file{featurefunction.h}.

Also provided:
\begin{itemize}
\item A function \code{computeFFMatrix()} which evaluates feature functions on
      all the source-target pairs and fills an array of matrices with the
      results, so that the array of matrices may be used as an argument to
      \code{powell()} or \code{linemax()}.
\item A function \code{writeFFMatrix()} used to write the values from an array
      of matrices to a file (no longer used - \code{FileFF} is better).
\item A function \code{readFFMatrix()} that will read the values from a file
      and fill in an array of matrices with the values (this function is the
      inverse of the previous function) (no longer used - \code{FileFF} is
      better).
  As certain feature function take a while to load and/or evaluate, these two
  functions are useful when running the program multiple times with the same
  data.
\item The class \code{FeatureFunctionSet}, representing an ordered set of
      feature functions.
\end{itemize}

The implementation of these functions is found in \file{featurefunction.cpp}.
They do not depend on any other code here.

The following feature functions have been implemented:
\begin{itemize}
\item \code{FileFF} takes values from a file; it is found in
      \file{featurefunction.\{h,cpp\}}.
\item \code{multiColumnFileFF} efficiently takes values in multiple columns
      from a file; it is found in \file{multiColumnFileFF.\{h,cc\}}.
\item \code{LengthFF} uses the number of words in the target sentence as its
      value; it is found in \file{featurefunction.\{h,cpp\}}.
\item In \file{ibm\_ff.h}, there are feature functions for the IBM Models 1 and
      2 log probabilities.  They are aptly named \code{IBM1TgtGivenSrc},
      \code{IBM1SrcGivenTgt}, \code{IBM2TgtGivenSrc} and
      \code{IBM2SrcGivenTgt}.  They use the respective IBM model forward and
      reverse translation log probabilities.
\item In \file{lm\_ff.\{h,cpp\}}, there is a feature function,
      \code{NgramFF}, that makes use of our language model module; in
      particular, the feature function's value is the N-gram language model
      score.
\item In \file{ibm1aaron.\{h,cc\}} is \code{IBM1WTransTgtGivenSrc}, defined as:
  \[ \argmax_{f:S \hookrightarrow T} \sum_{s \in S} \frac{P(f(s)|s)}{\max_{t
  \in \mathcal{T}} P(t|s)}, \] where $S$ is the set of source tokens, $T$ is
  the set of target tokens, and $\mathcal{T}$ is the set of all words in the
  target language.  The intention here is to estimate the number of source
  words that have good translations.  \code{IBM1WTransSrcGivenTgt} (found in
  the same files) computes the same thing but with source and target reversed.
\end{itemize}

Note that in the tm module there is a program that computes a phrase-based
translation model score.  This has not been implemented as a feature function
here because it produces two values (distortion and translation model score).

\section*{\file{rescoring\_general.h}}

General constants and useful macros and functions are defined here, and used by
all other code files.

\section*{Main Programs}

The rescoring module consists of four programs:
\begin{itemize}
\item \textbf{rescore\_train} is used to train a model of rescoring weights
      using Powell's algorithm.
\item \textbf{rescore\_translate} is used to perform rescoring with an existing
      model.
\item \textbf{rescore\_test} uses an existing model to rescore a test corpus
      and determine the BLEU score attained.
\item \textbf{gen\_feature\_values} can be used to generate all the values for
      a given feature function.
\end{itemize}

Naturally, the main function for each program is found in the respective
\file{.cc} file.

\section*{Running the Programs}

Here are the main steps for using this module:

\begin{enumerate}
\item
  Compile a source text, a list of nbest translations for each sentence in it,
  and a set of reference translations for each sentence in it. See below for
  the formats for these three files.

\item
  Decide on a set of features to use, eg language model, translation model,
  etc. For each feature, generate a score for each nbest list entry.  There are
  three ways to generate feature values: (1) using an external program; (2)
  using the program \textbf{gen\_feature\_values}; and (3) dynamically during
  model training.  The $2^{nd}$ and $3^{rd}$ methods use a canned set of
  features known to the rescoring module - see below for info on how to add new
  ones to this set.  The $3^{rd}$ method is useful mainly for simple features
  that can be calculated quickly, eg sentence length.  If you want to use an
  external program, see the format for feature value files below.

\item
  Train a model using the program \textbf{rescore\_train}.  This takes a model
  as input and produces a new one as output.  The input model lists the set of
  features to use and supplies an optional initial weight for each; the output
  model gives final weights for each feature. See below for model file format.
  The other parameters to \textbf{rescore\_train} are source, nbest, and
  reference files.  Use rescore\_train -h for more specific info.

\item
  Test a model using the program \textbf{rescore\_test}.  This takes a trained
  model as input, and generates a BLEU score for a given source, nbest,
  reference triple (which should be different from that used to produce the
  model).  Use rescore\_train -h for more specific info.

\item
  Translate, using the program \textbf{rescore\_translate}.  This takes a
  trained model as input and produces a best translation for a given source
  text and nbest list.  Use rescore\_translate -h for more specific info.

\end{enumerate}

\section*{File Formats}

\begin{itemize}

\item \textbf{Source, nbest, and refs files}\\
  All of these contain one segment per line, with tokens separated by
  whitespace.  Each source line corresponds to a block of lines in the nbest
  file, and a block of lines in the refs file.  Blocks are in order, so that
  the ith source sentence corresponds to the ith block.  They are also fixed
  size and contiguous, so for example if there are S source sentences and K
  entries in each nbest list, then there will be SK lines in the nbest file.
  Similarly if there are R reference translations per source sentence, there
  will be SR lines in the refs file.  If your data has variable-sized blocks,
  pad short blocks with blank lines.

\item \textbf{Model file}\\
  This contains a set of entries, one per line, in the form:\\
  feature\_name[:arg] [weight]\\ For example, an IBM1-based feature correspond
  to the line:\\ IBM1SrcGivenTarget:ttable-file-location 2.345 An important
  feature is FileFF, which designates a set of feature values stored in a
  feature file.

  For example, to use the contents of a file named very-important-feature, you
  would include this line:\\ FileFF:very-important-feature [weight] A list of
  all features known to the module is in the create() function in
  \file{featurefunction.cpp}.

\item \textbf{Feature value file}\\
  This is simply a column of numbers that matches 1-1 with the nbest file it
  pertains to.  Individual features are scalar-valued, so there can only be one
  number per line.

\item \textbf{Multi-column value file}\\
  Just like a feature value file, except each line should contains exactly N
  numbers if the file encode N features.

\end{itemize}

\section*{Adding New Features}
To add to the set of known features:

\begin{enumerate}
  
\item Write a class derived from the \code{FeatureFunction} interface in
  \file{featurefunction.h}.  There are some examples of derived classes in
  \file{featurefunction.h} (but yours doesn't have to be put there, of course).
  The only thing feature functions have to do is return a value when called
  with a source/target segment pair.  There are also functions to inform them
  of the wider context (the whole source text under translation, and the whole
  current nbest list), but these don't have to be instantiated.
  
\item Add a clause to the \code{FeatureFunctionSet::create()} function, defined
  in \file{featurefunction.cpp}.  This should just check for a match with your
  feature's name and call the corresponding constructor.  Note that you must be
  able to construct your class from a single string argument - this is the
  "arg" string that is specified in the model file.

\item
  Modify the Makefile to link with any necessary module-external libraries.

\item
  Test using \textbf{gen\_feature\_values}, please!

\end{enumerate}

If you just want to try out a feature function, note that it may be easier to
write the values to a file and use FileFF rather than adding the function to
the known set.

\bibliographystyle{plain} \bibliography{rescoring_overview}

\end{document}
